%!TEX TS-program = xelatex

% Author: Amet Umerov (admin@amet13.name)
% https://github.com/Amet13/bachelor-diploma
% Tuned for MAI by Rodin Fedor
% Forked by you

% НАСТРОЙКИ ФОРМАТИРОВАНИЯ

\def \EmptyPageAfterTitle {}			% закомментировать, если не нужна пустая страница после титульной
% \def \TwoSided {}						% если не закомментировано, то поля слева и справа будут чередоваться, нужно для двухсторонней печати. Если предполагается односторонняя печать, закомментируйте.

% НАСТРОЙКИ ИМЕН, УЧЕНЫХ СТЕПЕНЕЙ И Т.П.

\newcommand{\StudentFioLastname}{Гунин}					% фамилия студента
\newcommand{\StudentFioFirstname}{Филипп}				% имя студента
\newcommand{\StudentFioSurname}{Алексеевич}				% отчество студента

\newcommand{\StudentKurs}{4}							% курс обучения студента
\newcommand{\StudentGroup}{М2О-404С-19}					% группа студента

\newcommand{\PrepodFioLastname}{Кожевников}				% фамилия преподавателя
\newcommand{\PrepodFioFirstname}{Глеб}					% имя преподавателя
\newcommand{\PrepodFioSurname}{Денисович}				% отчество преподавателя

\newcommand{\PrepodCaptionFirst}{преподаватель}		% первая строчка ученой степени преподавателя
\newcommand{\PrepodCaptionSecond}{}		% вторая строчка ученой степени преподавателя


% НАСТРОЙКИ НАЗВАНИЙ РАБОТЫ

% тип работы (отчет, курсовая работа и т.д.)
\newcommand{\RabotaType}{РАСЧЕТНАЯ РАБОТА}

% о чем? (строчка после типа работы)
\newcommand{\RabotaOChem}{по дисциплине «Проектирование технологических процессов»}

% тема или название работы
\newcommand{\RabotaThemeName}{\pdftitleP}


% НАСТРОЙКИ ИНСТИТУТА И КАФЕДРЫ

% номер института
\newcommand{\InstitutNumber}{2}

% название института без кавычек
\newcommand{\InstitutName}{Авиационные, ракетные двигатели и энергетические установки}

% номер кафедры
\newcommand{\KafedraNumber}{205}

% название кафедры без кавычек
\newcommand{\KafedraName}{Технология производства двигателей летательных аппаратов}


\ifx \TwoSided \undefined
	\documentclass[a4paper,titlepage,14pt]{extarticle}
\else
	\documentclass[a4paper,twoside,titlepage,14pt]{extarticle}
\fi

\newcommand{\pdfautorP}{Филипп Гунин} 
\newcommand{\pdftitleP}{Анализ динамики процессов в гидромагистрали с учетом внешних колебаний и работы демпфера} % на тему


% ИНДЕКСЫ
\newcommand{\ig}{\text{г}}
\newcommand{\id}{\text{д}}

% РАЗМЕРНОСТИ
\newcommand{\dmm}{\text{  мм}}
\newcommand{\dsec}{\text{  сек}}
\newcommand{\dmin}{\text{  мин}}
\newcommand{\dkg}{\text{  кг}}
\newcommand{\dkgps}{\ \ \frac{\text{кг}}{\text{с}}} % Подключаем список параметров
\input{inc/preamble} % Подключаем преамбулу
\usepackage{tabularx}
\usepackage{makecell}

%%% Начало документа
\begin{document}

% НАЧАЛО ТИТУЛЬНОГО ЛИСТА
\begin{titlepage}

\begin{center}
	\linespread{1.4}

	% Логотип МАИ
	% \begin{figure}[h!]
	% 	\includegraphics[scale=0.25]{logo.png}
	% \end{figure}

	\normalsize{Министерство науки и высшего образования\\ Российской Федерации}\\
	\vspace{0.25cm}
	\normalsize{Федеральное государственное бюджетное образовательное\\ учреждение высшего образования}\\
	\vspace{0.25cm}
	\normalsize\textbf{«МОСКОВСКИЙ АВИАЦИОННЫЙ ИНСТИТУТ»}\\ {(НАЦИОНАЛЬНЫЙ ИССЛЕДОВАТЕЛЬСКИЙ УНИВЕРСИТЕТ)}\\
	\noindent\rule{\textwidth}{0.4pt} \\ \vspace{0.25cm}
	\normalsize
	{Институт № \InstitutNumber \ «\InstitutName»\\ Кафедра \KafedraNumber \ «\KafedraName»}\\
	\vfill
	
	{\RabotaType}\\
	{\RabotaOChem}\\
	\hfill\break	
	{\RabotaThemeName}\\
	
	\vfill
	
	\begin{tabularx}{0.95\textwidth}{ l X l l }
		
		Выполнил & \makecell[c]{\underline{\hspace{3cm}}} & \makecell[l]{студент \StudentKurs \ курса \\ группы \StudentGroup} & \makecell[r]{\StudentFioLastname \ \StudentFioFirstname \\ \StudentFioSurname} \\
		
		\multicolumn{4}{ c }{ } \\
		
		Проверил & \makecell[c]{\underline{\hspace{3cm}}} & \makecell[l]{\PrepodCaptionFirst \\ \PrepodCaptionSecond} & \makecell[r]{\PrepodFioLastname \ \PrepodFioFirstname \\ \PrepodFioSurname} \\
		
	\end{tabularx}

\end{center}
\hfill \break
\begin{center} Москва \the\year{} \end{center}
% \begin{center} Москва 2022 \end{center}
\thispagestyle{empty} % выключаем отображение номера для этой страницы
\end{titlepage}

\ifx \EmptyPageAfterTitle \undefined
\else
	\newpage
	\thispagestyle{empty}
	\mbox{}
	\newpage
\fi

%\newpagel

% КОНЕЦ ТИТУЛЬНОГО ЛИСТА


% СОДЕРЖАНИЕ 
\tableofcontents 
\clearpage

\anonsection{З\uppercase{адание}}

Основные этапы выполнения работы:
\begin{enumerate}
    \item[1.] Создать 3D модель детали.  
    \item[2.] Оформить рабочий чертёж детали согласно требованием ЕСКД.
    \item[3.] Рассмотреть химический состав физико-механических свойств обрабатываемого материала, область его применения. Охарактеризовать обрабатываемость материала резаньем. 
    \item[4.] Выбрать метод изготовления заготовки и спроектировать её. 
    \item[5.] Составить план обработки детали.  
    % \item[6.] Определить технологические базы. 
    \item[6.] Выбрать необходимое металлорежущее оборудование и дать его краткую характеристику. 
    \item[7.] Рассчитать оптимальный режим резания.
    \item[8.] Составить маршрутную карту, операционную карту.    
\end{enumerate}

% \clearpage
\section{Трёхмерная модель детали}

Трёхмерная модель детали выполнена в програмном комплексе САПР Solidwors 2022.
Модель изображена на рисунке \ref{pic:supernigger}.

\addimghere{inc/images/model.png}{}{3D модель детали}{pic:supernigger}




% \clearpage
\section{Рабочий чертёж детали}

Основные размеры представляют собой:
\begin{itemize}
    \item Длина всей детали - $L=60$ мм
    \item Внешний диаметр - $D=150$ мм
    \item Диаметры отверстий - $d_1=50,\ d_2=40,\ d_3=16$ мм
\end{itemize}


Чертёж детали приведён в приложении А.


% \clearpage
\section{Характеристика обрабатываемого материала}

\subsection{Химический состав}

Обрабатываемый материал – Сталь-45. Химический состав данного материала в процентном соотношении в соответствии с ГОСТ 1050-2013 приведен в таблице \ref{tab:nigger}.

\begin{table}[H]
    \centering
    \caption[]{Химический состав материала Сталь-45 в процентном соотношении (ГОСТ 1050-2013)}
    \label{tab:nigger}
    \begin{tabular}{|p{1.5cm}|p{1.5cm}|p{1.5cm}|p{1.5cm}|p{1.5cm}|p{1.5cm}|p{1.5cm}|p{1.5cm}|}
        \hline \textbf{$C$} & \textbf{$Mn$} & \textbf{$Si$} & \textbf{$Ni$} & \textbf{$Cr$} & \textbf{$Cu$} & \textbf{$S$} & \textbf{$P$} \\
        \hline 0,42—0,5\%   & 0,5—0,8\%     & 0,17—0,37\%   & до 0,3\%      & до 0,25\%     & до 0,3\%      & до 0,035\%    & до 0,03\% \\
        \hline
    \end{tabular}
\end{table}


\subsection{Физико-механические свойства}

Детали из стали марки 45 подвергаются нормализации при температуре 860-880° С или закалке в воде с температуры 840-860° С с последующим отпуском.\cite{site-stal-45}

Сталь горячекатаная согласно ГОСТ 1050-88 имеет предел прочности $\sigma_\textit{В}=600$ МПа с относительным удлинением $\delta=16\%$.


\subsection{Область применения}

Сталь-45 применяется для изготовления таких конструкционных элементов, как вал-шестерни, коленчатые и распределительные валы, шестерни, шпиндели, бандажи, цилиндры, кулачки и другие нормализованные, улучшаемые и подвергаемые поверхностной термообработке детали, от которых требуется повышенная прочность.\cite{site-stal-45}


\subsection{Технологические свойства}

Обрабатываемость резанием - $K_\textit{v тв.спл} = 1$ и $K_\textit{v б.ст} = 1$ в горячекатаном состоянии при $H\!B = 170 \div 179$ и $\sigma_\textit{в} = 640$ МПа.


% \clearpage
\section{Метод изготовления заготовки}

Проектировка заготовки из материала Сталь-45 для изготовления детали осуществляется по следующему плану. 

Деталь получается путем токарной обработки по всему контуру, а затем – фрезерной обработки. Необрабатываемых поверхностей в детали нет. По конфигурации деталь имеет цилиндрическую форму. В этой связи наиболее удобный метод получения заготовки из круглого горячекатаного прутка (ГОСТ 2590-2006). Выбираем прокат обычной точности.

Диаметр описанной окружности детали $d = 150$ мм, тогда предельные отклонения для детали будут:

\begin{equation}
    es_\text{д} = 0    \text{ мм}
\end{equation}
\begin{equation}
    ei_\text{д} = -1    \text{ мм}
\end{equation}

Наименьший операционный припуск $Z_{min}$ складывается из отдельных элементов:
\begin{equation}
    Z_\text{min} = Z_1 + Z_2 + Z_3 = 1.35   \text{,}
\end{equation}
где $Z_1$ - слой материала, который необходимо удалить с заготовки для устранения неровностей $R_z$ и дефектного слоя $h$:
\begin{equation}
    Z_1 = R_z + h = 150 + 500 = 650 \text{ мкм,}
\end{equation}
$Z_2 = 0.6 \text{ мм}$ - слой, удаляемый для компенсации погрешностей формы
Номинальный размер заготовки, $Z_3 = 0.1 \text{ мм}$ - слой, удаляемый для компенсации погрешностей установки.\cite{metodika}

Диаметр заготовки \cite{metodika}:
\begin{equation}
    d_\text{з} = d_\text{д} + 2 \cdot Z_{min} + 2 \cdot ei_\text{д} = 150 + 2 \cdot 1.35 + 2 \cdot (-1) = 154.7    \text{ мм}
\end{equation}

Ближайшее значение диаметра прутка по ГОСТ 2590-2006:
\begin{equation}
    d_\text{з} =  155^{+0.8}_{-2.0}   \text{ мм}
\end{equation}

Номинальный припуск на обработку \cite{metodika}:
\begin{equation}
    Z_\text{ном} = \dfrac{d_\text{з} - d_\text{д}}{2} = \dfrac{155 - 150}{2} = 2.5   \text{ мм}
\end{equation}

Максимальный припуск на обработку \cite{metodika}:
\begin{equation}
    Z_\text{max} = Z_\text{min} + ei_\text{д} + es_\text{з} = 2.5 + 1 + 0.8 = 4.3   \text{ мм}
\end{equation}

Припуск по торцам равен номинальному припуску на диаметр, тогда длина заготовки \cite{metodika}:
\begin{equation}
    L = L_\text{д} + 2 \cdot Z_\text{ном} = 60 + 2 \cdot 2.5 = 65   \text{ мм}
\end{equation}

% \clearpage
\section{План обработки детали}


\begin{longtable}{|p{1.5cm}|p{9cm}|p{5cm}|}
    \caption{План обработки детали} \\
    \hline №   & Эскиз операций                       & Обработка, операция, инструмент \\
    \endhead
    \hline 005 & \addimganon{inc/images/planNew/000}{0.6} & Отрезать заготовку и установить в трехкулачковом патроне, ножовка по металлу. \\
    \hline 010 & \addimganon{inc/images/planNew/00}{0.6} & Токарная, подрезание торца, проходной отогнутый резец \\
    \hline 010 & \addimganon{inc/images/planNew/01}{0.6} & Токарная, точение по верху, проходной упорный резец \\
    \hline 015 & \addimganon{inc/images/planNew/02}{0.6} & Токарная, наружнее протачивание в упор на $\varnothing 70$ мм на длину 30 мм, проходной упорный резец \\
    \hline 020 & \addimganon{inc/images/planNew/03}{0.6} & Токарная, сверление сквозного отверстия, сверло $\varnothing 22$ мм спиральное с коническим хвостиком \\
    \hline 020 & \addimganon{inc/images/planNew/04}{0.6} & Токарная, рассверление сквозного отверстия, сверло $\varnothing 36$ мм спиральное с коническим хвостиком \\
    \hline 020 & \addimganon{inc/images/planNew/05}{0.6} & Токарная, наружнее протачивание в упор на $\varnothing 60$ мм на длину 20 мм, проходной упорный резец \\
    \hline 020 & \addimganon{inc/images/planNew/06}{0.6} & Токарная, растачивание отверстия до $\varnothing 40$ мм, резец расточной для сквозных отверстий \\
    \hline 020 & \addimganon{inc/images/planNew/07}{0.6} & Токарная, растачивание отверстия в упор на $\varnothing 50$ мм на длину 50 мм, резец расточной для обработки глухих отверстий \\
    \hline 025 & \addimganon{inc/images/planNew/08}{0.7} & Фрезерная, фрезерование двух отверстий, фреза концевая $\varnothing 16$ мм \\
    \hline 025 & \addimganon{inc/images/planNew/09}{0.7} & Фрезерная, фрезерование ромба, фреза цилиндрическая \\
    \hline 025 & \addimganon{inc/images/planNew/10}{0.7} & Фрезерная, фрезерование скруглений, фреза цилиндрическая \\
    \hline
\end{longtable}


% \clearpage
\section{Расчет режима резания}


\subsection{Операция сверлильная}



\subsubsection{Исходные данные}

% Исходные данные приведены в таблице \ref{tab:61}

% \begin{tabular}
%     \label{tab:61}
    \begin{longtable}{|c|c|}
        \caption{Исходные данные} \\
        \hline Параметр & Значение \\
        \endhead
        \hline Отверстие & Сквозное \\
        \hline Материал & Сталь 45 \\
        \hline Предел прочности, $\sigma_{\text{в}}$ & 600, МПа \\
        \hline Твердость, $HB$ & 170, МПа \\
        \hline Диаметр отверстия, $D$ & 36, мм \\
        \hline Длина отверстия, $L$ & 60, мм \\
        \hline
    \end{longtable}
% \end{tabular}




\subsubsection{Режущий инструмент и режим сверления}

Так как диаметр отверстия больше $25$ мм, то сверление проводится в два подхода. Для обработки сквозного отверстия $D=36H14$ и шероховатостью $Rz=20$ на токарно-винторезном станке 16к20 применим операции сверления и рассверливания. Диаметр отверстия после сверления и рассверливания:

\begin{equation}
	d = D = 36 \text{ мм}
\end{equation}



\subsubsection{Сверление}

Для сверления выбирается спиральное сверло диаметра
\begin{equation}
	d\sv = d \cdot 0.6 = 21.59 \to 22 \dmm .
\end{equation}

Материал сверла $P18$ \cite{2}. Определяется глубина резания
\begin{equation}
	t\sv = 0.5 \cdot d\sv = 11 \dmm ,
\end{equation}

подача \cite{1}, скорректированная по паспорту станка \cite{3}
\begin{equation}
	S\sv = 0.38 \div 0.43 \to 0.4 \dmm ,
\end{equation}

скорость резания \cite{1}:
\begin{equation}
	V\sv = \frac{C_v \cdot d\sv^{q}}{T\sv^{m}S\sv^{y}} \cdot K_v = \frac{9.8 \cdot 22^{0.4}}{50^{0.2} \cdot 0.4^{0.5}} \cdot 0.347 = 24.39 \text{  м/мин}.
\end{equation}

Частота вращения шпинделя будет равна
\begin{equation}
	n\sv = \frac{1000 \cdot V\sv}{\pi d\sv} = 353.2 \text{  об/мин},
\end{equation}
скорректировав по паспортным данным станка, получим $n\sv = 400$ об/мин.

Действительная скорость резца тогда будет 
\begin{equation}
	V\sv = \frac{\pi \cdot d\sv \cdot n\sv}{1000} = 27.6 \text{  м/мин} .
\end{equation}

Крутящий момент вычисляем по формуле
\begin{equation}
	M_\text{кр} = 10 \cdot C_m \cdot d\sv^q \cdot S\sv^y \cdot K_p = 10 \cdot 0.0345 \cdot 22^{2} \cdot 0.4^{0.7} \cdot 0.885 = 67 \text{  Н*м} .
\end{equation}

Определяем осевую силу
\begin{equation}
	Po\sv = 10 \cdot C_p \cdot d\sv^q \cdot S\sv^y \cdot K_p = 10 \cdot 68 \cdot 22^{1} \cdot 0.4^{0.7} \cdot 0.885 = 6663 \text{  Н},
\end{equation}
мощность резания
\begin{equation}
	Np\sv = \frac{M_\text{кр} \cdot n\sv}{9750} = 2.78 \text{  кВт}.
\end{equation}

При мощности станка $N_\text{э} = N_\text{д} \cdot \eta = 10 \cdot 0.75 = 7.5$ кВт операция осуществима. Определяем время обработки $l_1 = 8$
\begin{equation}
    T\sv = \frac{L + l_1}{S\sv \cdot n\sv} = 0.425 \dmin.
\end{equation}




\subsubsection{Рассверливание}

Для рассверливания выбирается спиральное сверло диаметра
\begin{equation}
	d\rv = D = 36 \dmm .
\end{equation}

Материал сверла $P18$ \cite{2}. Определяется глубина резания
\begin{equation}
	t\rv = 0.5 \cdot (d\rv - d\sv) = 7 \dmm ,
\end{equation}

подача \cite{1}, скорректированная по паспорту станка \cite{3}
\begin{equation}
	S\rv = 0.48 \div 0.58 \to 0.5 \dmm ,
\end{equation}

скорость резания \cite{1}:
\begin{equation}
	V\rv = \frac{C_v \cdot d\rv^{q}}{T\rv^{m}S\rv^{y}} \cdot K_v = \frac{16.2 \cdot 36^{0.4}}{70^{0.2} \cdot 0.5^{0.5}} \cdot 0.347 = 27.8 \text{  м/мин}.
\end{equation}

Частота вращения шпинделя будет равна
\begin{equation}
	n\rv = \frac{1000 \cdot V\rv}{\pi d\rv} = 246.2 \text{  об/мин},
\end{equation}
скорректировав по паспортным данным станка, получим $n\rv = 250$ об/мин.

Действительная скорость резца тогда будет 
\begin{equation}
	V\rv = \frac{\pi \cdot d\rv \cdot n\rv}{1000} = 28.26 \text{  м/мин} .
\end{equation}

Крутящий момент вычисляем по формуле
\begin{equation}
	M_\text{кр} = 10 \cdot C_m \cdot d\rv^q \cdot t\rv^x \cdot S\rv^y \cdot K_p = 
                10 \cdot 0.09 \cdot 36^{1} \cdot 7^{0.9} \cdot 0.5^{0.8} \cdot 0.845 = 90.7 \text{  Н*м} .
\end{equation}

Определяем осевую силу
\begin{equation}
	Po\rv = 10 \cdot C_p \cdot t\rv^x \cdot S\rv^y \cdot K_p = 
            10 \cdot 67 \cdot 7^{1.2} \cdot 0.5^{0.65} \cdot 0.885 = 3731 \text{  Н},
\end{equation}
мощность резания
\begin{equation}
	Np\rv = \frac{M_\text{кр} \cdot n\rv}{9750} = 2.32 \text{  кВт}.
\end{equation}

При мощности станка $N_\text{э} = N_\text{д} \cdot \eta = 10 \cdot 0.75 = 7.5$ кВт операция осуществима. Определяем время обработки $l_1 = 0.6$
\begin{equation}
    T\rv = \frac{L + l_1}{S\rv \cdot n\rv} = 0.484 \dmin.
\end{equation}

Общее технологическое время
\begin{equation}
	T = T\sv + T\rv = 0.9 \dmin.
\end{equation}


% \clearpage
\vspace*{\fill}
% \centering{\uppercase{Приложение}}
\section{Маршрутная карта}
\vspace*{\fill}

\eject \pdfpagewidth=297mm \pdfpageheight=210mm
\newgeometry{
    layoutwidth=297mm,
    layoutheight=210mm,
    top=-0.1mm,
    left=-10mm,
    right=-1mm,
    bottom=-1mm
}
\includegraphics*{inc/images/MK/1}
\newpage{\includegraphics*{inc/images/MK/2}}

\eject \pdfpagewidth=210mm \pdfpageheight=297mm
\restoregeometry


% ---------------------------------------------------------------------------
\vspace*{\fill}
% \centering{\uppercase{Приложение}}
\section{Операционная карта}
\vspace*{\fill}

\eject \pdfpagewidth=297mm \pdfpageheight=210mm
\newgeometry{
    layoutwidth=297mm,
    layoutheight=210mm,
    top=-0.1mm,
    left=-10mm,
    right=-1mm,
    bottom=-1mm
}
\includegraphics*{inc/images/OK/01}
\newpage{\includegraphics*{inc/images/OK/02}}
\newpage{\includegraphics*{inc/images/OK/03}}
\newpage{\includegraphics*{inc/images/OK/04}}
\newpage{\includegraphics*{inc/images/OK/05}}
\newpage{\includegraphics*{inc/images/OK/06}}
\newpage{\includegraphics*{inc/images/OK/07}}
\newpage{\includegraphics*{inc/images/OK/08}}
\newpage{\includegraphics*{inc/images/OK/09}}
\newpage{\includegraphics*{inc/images/OK/10}}
\newpage{\includegraphics*{inc/images/OK/11}}
\newpage{\includegraphics*{inc/images/OK/12}}
\newpage{\includegraphics*{inc/images/OK/13}}
\newpage{\includegraphics*{inc/images/OK/14}}


\eject \pdfpagewidth=210mm \pdfpageheight=297mm
\restoregeometry

\begingroup 
\renewcommand{\section}[2]{\anonsection{\uppercase{Библиографический список}}}
\begin{thebibliography}{00}

\bibitem{metodika}
    В.Н. Б\lowercase{ОТЯШИН}
    С.И. Т\lowercase{АРАСОВ}
    Т\lowercase{ЕХНОЛОГИЯ} О\lowercase{БРАБОТКИ} М\lowercase{АТЕРИАЛОВ}, П\lowercase{РИМЕНЯЕМЫХ} в Э\lowercase{НЕРГОМАШИНОСТРОЕНИИ}: Учебное пособие. - М.: Изд-во МАИ, 2006 - 96 с.

\bibitem{site-stal-45}
    Сталь 45 конструкционная углеродистая качественная
    [Электронный ресурс] // 
    URL: https://enginiger.ru/materials/uglerodistye-stali/stal-45-konstruktsionnaya-uglerodistaya-kachestvennaya/
    (дата обращения: 10.апреля.2023)

\bibitem{3}
    Паспортные данные станка токарного 16к20

\bibitem{2}
    Справочник нормировщика–машиностроителя. В 2-х т. Т. 2 Под редакцией Е. И. Стружестраха.- Техническое нормирование станочных работ: Государственное научно-техническое издательство машиностроительной литературы, Москва, 1961 - 891 с.

\bibitem{1}
    Cправочник технолога-машиностроителя. В 2-х т. C74 T. 2 Под ред. А. Г. Косиловой и Р. К. Мещерякова.- 4-е изд., перераб. и доп.- M.: Mашиностроение, 1986. 496 с., ил.

\end{thebibliography}
\endgroup

\clearpage


% ПРИЛОЖЕНИЯ
\vspace*{\fill}
\centering{\uppercase{Приложение}}
\vspace*{\fill}

\clearpage

\appsection{Приложение}
\centering{\uppercase{Програмный код решения}}
\vspace{\baselineskip}

% \setcounter{section}{0}
% \addtocontents{toc}{\protect\setcounter{tocdepth}{-1}}
 
% \includepdf{images/drawing1.pdf}


% \hfill \break
% \begin{center}
% 	\LaTeX
% \end{center}

\end{document}
%%% Конец документа