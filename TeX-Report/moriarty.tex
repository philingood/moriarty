%!TEX TS-program = xelatex

% Author: Amet Umerov (admin@amet13.name)
% https://github.com/Amet13/bachelor-diploma
% Tuned for MAI by Rodin Fedor
% Forked by you

% НАСТРОЙКИ ФОРМАТИРОВАНИЯ %%%%%%%%%%%%%%%%%%%%%%%%%%%%%%%%%%%%%%%%%%%%%%%%%%%%%%%%%%%%%%%

\def \EmptyPageAfterTitle {}			% закомментировать, если не нужна пустая страница после титульной
% \def \TwoSided {}						% если не закомментировано, то поля слева и справа будут чередоваться, нужно для двухсторонней печати. Если предполагается односторонняя печать, закомментируйте.

% НАСТРОЙКИ ИМЕН, УЧЕНЫХ СТЕПЕНЕЙ И Т.П. %%%%%%%%%%%%%%%%%%%%%%%%%%%%%%%%%%%%%%%%%%%%%%%%%

\newcommand{\StudentFioLastname}{Гунин}					% фамилия студента
\newcommand{\StudentFioFirstname}{Филипп}				% имя студента
\newcommand{\StudentFioSurname}{Алексеевич}				% отчество студента

\newcommand{\StudentKurs}{5}							% курс обучения студента
\newcommand{\StudentGroup}{М2О-504С-19}					% группа студента

\newcommand{\PrepodFioLastname}{Мартиросов}				% фамилия преподавателя
\newcommand{\PrepodFioFirstname}{Давид}					% имя преподавателя
\newcommand{\PrepodFioSurname}{Суренович}				% отчество преподавателя

\newcommand{\PrepodCaptionFirst}{Доктор технических наук}		% первая строчка ученой степени преподавателя
\newcommand{\PrepodCaptionSecond}{Профессор}		% вторая строчка ученой степени преподавателя


% НАСТРОЙКИ НАЗВАНИЙ РАБОТЫ %%%%%%%%%%%%%%%%%%%%%%%%%%%%%%%%%%%%%%%%%%%%%%%%%%%%%%%%%%%%%
% тип работы (отчет, курсовая работа и т.д.)
\newcommand{\RabotaType}{КУРСОВАЯ РАБОТА}
% о чем? (строчка после типа работы)
\newcommand{\RabotaOChem}{по дисциплине «Автоматика и регулирование ЖРДУ»}
% тема или название работы
\newcommand{\RabotaThemeName}{\pdftitleP}


% НАСТРОЙКИ ИНСТИТУТА И КАФЕДРЫ %%%%%%%%%%%%%%%%%%%%%%%%%%%%%%%%%%%%%%%%%%%%%%%%%%%%%%%%%
% номер института
\newcommand{\InstitutNumber}{2}
% название института без кавычек
\newcommand{\InstitutName}{Авиационные, ракетные двигатели и энергетические установки}
% номер кафедры
\newcommand{\KafedraNumber}{202}
% название кафедры без кавычек
\newcommand{\KafedraName}{Ракетные двигатели}
%%%%%%%%%%%%%%%%%%%%%%%%%%%%%%%%%%%%%%%%%%%%%%%%%%%%%%%%%%%%%%%%%%%%%%%%%%%%%%%%%%%%%%%%
%%%%%%%%%%%%%%%%%%%%%%%%%%%%%%%%%%%%%%%%%%%%%%%%%%%%%%%%%%%%%%%%%%%%%%%%%%%%%%%%%%%%%%%%


\ifx \TwoSided \undefined
	\documentclass[a4paper,titlepage,14pt]{extarticle}
\else
	\documentclass[a4paper,twoside,titlepage,14pt]{extarticle}
\fi

\newcommand{\pdfautorP}{Филипп Гунин} 
\newcommand{\pdftitleP}{Разработка документации технологических процессов \\ Вариант 3} % на тему



% Индексы
\newcommand{\sv}{_{\text{св}}}
\newcommand{\rv}{_{\text{рв}}}  

% РАЗМЕРНОСТИ
\newcommand{\dmm}{\text{  мм}}
\newcommand{\dsec}{\text{  сек}}
\newcommand{\dmin}{\text{  мин}} % Подключаем список параметров
%%% Преамбула %%%

\usepackage{fontspec}  % XeTeX
\usepackage{xunicode}  % Unicode для XeTeX
\usepackage{xltxtra}   % Верхние и нижние индексы
\usepackage{pdfpages}  % Вставка PDF
% \usepackage{lscape}   % Поворот текста в альбомный вид
\usepackage{pdflscape} % Поворот текста в альбомный вид
\usepackage{longtable} % Длинные таблицы
% \usepackage[paper=A4]{typearea} % Изменение размера страниц
% \usepackage{typearea} % Изменение размера страниц

\usepackage{listings} % Оформление исходного кода
\lstset{
    basicstyle=\small\ttfamily, % Размер и тип шрифта
    breaklines=true, % Перенос строк
    tabsize=2, % Размер табуляции
    literate={--}{{-{}-}}2 % Корректно отображать двойной дефис
}

% \makeatletter
% \renewcommand{\@seccntformat}[1]{}
% \makeatother

% Шрифты, xelatex
\defaultfontfeatures{Ligatures=TeX}
\setmainfont{Times New Roman} % Нормоконтроллеры хотят именно его
\newfontfamily\cyrillicfont{Times New Roman}
%\setsansfont{Liberation Sans} % Тут я его не использую, но если пригодится
%\usepackage{courier}
\newfontfamily{\cyrillicfonttt}{Courier New}
% \setmonofont{Courier} % Моноширинный шрифт для оформления кода

% Русский язык
\usepackage{amssymb,amsfonts,amsmath} % Математика
\usepackage[T2A]{fontenc}
% \usepackage{unicode-math}
% \setmathfont{texgyrepagella-math.otf}
\usepackage{polyglossia}
\setdefaultlanguage{russian}

\numberwithin{equation}{section} % Формула вида секция.номер

\usepackage{enumerate} % Тонкая настройка списков
\usepackage{indentfirst} % Красная строка после заголовка
\usepackage{float} % Расширенное управление плавающими объектами
\usepackage{multirow} % Сложные таблицы

% Вставка изображений
\usepackage{graphicx} % Вставка картинок и дополнений
\usepackage{wrapfig} % Обтекание картинок текстом
\graphicspath{{images/}{images/userguide/}{images/testing/}{images/infrastructure/}{extra/}{extra/drafts/}}

% Формат подрисуночных записей
\usepackage{chngcntr}
\counterwithin{figure}{section}

% Гиперссылки
\usepackage{hyperref}
\hypersetup{
    colorlinks, urlcolor={black}, % Все ссылки черного цвета, кликабельные
    linkcolor={black}, citecolor={black}, filecolor={black},
    pdfauthor={\pdfautorP},
    pdftitle={\pdftitleP}
}

% Оформление библиографии и подрисуночных записей через точку
\makeatletter
\renewcommand*{\@biblabel}[1]{\hfill#1.}
\renewcommand*\l@section{\@dottedtocline{1}{1em}{1em}}
\renewcommand{\thefigure}{\thesection.\arabic{figure}} % Формат рисунка секция.номер
\renewcommand{\thetable}{\thesection.\arabic{table}} % Формат таблицы секция.номер
\def\redeflsection{\def\l@section{\@dottedtocline{1}{0em}{10em}}}
\makeatother

\renewcommand{\baselinestretch}{1.4} % Полуторный межстрочный интервал
\parindent 1.27cm % Абзацный отступ

\sloppy             % Избавляемся от переполнений
\hyphenpenalty=1000 % Частота переносов
\clubpenalty=10000  % Запрещаем разрыв страницы после первой строки абзаца
\widowpenalty=10000 % Запрещаем разрыв страницы после последней строки абзаца

% Отступы у страниц
\usepackage{geometry}
\geometry{left=3cm}
\geometry{right=1cm}
\geometry{top=2cm}
\geometry{bottom=2cm}

% Списки
\usepackage{enumitem}
\setlist[enumerate,itemize]{leftmargin=12.7mm} % Отступы в списках

\makeatletter
    \AddEnumerateCounter{\asbuk}{\@asbuk}{м)}
\makeatother
\setlist{nolistsep} % Нет отступов между пунктами списка
\renewcommand{\labelitemi}{--} % Маркет списка --
\renewcommand{\labelenumi}{\asbuk{enumi})} % Список второго уровня
\renewcommand{\labelenumii}{\arabic{enumii})} % Список третьего уровня

% Содержание
\usepackage{tocloft}
\renewcommand{\cfttoctitlefont}{\hspace{0.38\textwidth}\MakeTextUppercase} % СОДЕРЖАНИЕ
\renewcommand{\cftsecfont}{\hspace{0pt}}            % Имена секций в содержании не жирным шрифтом
\renewcommand\cftsecleader{\cftdotfill{\cftdotsep}} % Точки для секций в содержании
\renewcommand\cftsecpagefont{\mdseries}             % Номера страниц не жирные
\setcounter{tocdepth}{3}                            % Глубина оглавления, 3 = до subsubsection
% \setcounter{secnumdepth}{0}                         % Оглаление без нумерации

% Нумерация страниц справа сверху
%\usepackage{fancyhdr}
%\pagestyle{fancy}
%\fancyhf{}
%\fancyhead[R]{\textrm{\thepage}}
%\fancyheadoffset{0mm}
%\fancyfootoffset{0mm}
%\setlength{\headheight}{17pt}
%\renewcommand{\headrulewidth}{0pt}
%\renewcommand{\footrulewidth}{0pt}
%\fancypagestyle{plain}{ 
%    \fancyhf{}
%    \rhead{\thepage}
%}

% Формат подрисуночных надписей
\RequirePackage{caption}
\DeclareCaptionLabelSeparator{defffis}{ -- } % Разделитель
\captionsetup[figure]{justification=centering, labelsep=defffis, format=plain} % Подпись рисунка по центру
\captionsetup[table]{justification=raggedright, labelsep=defffis, format=plain, singlelinecheck=false} % Подпись таблицы слева
\addto\captionsrussian{\renewcommand{\figurename}{Рис.}} % Имя фигуры

% Пользовательские функции
\newcommand{\addimganon}[2]{ % Добавление одного рисунка
    % \begin{figure}
        % \centering
        \begin{center}
            \includegraphics[width=#2\linewidth]{#1}
        \end{center}
        % \caption{#3} \label{#4}
    % \end{figure}
}
\newcommand{\addimg}[4]{ % Добавление одного рисунка
    \begin{figure}
        \centering
            \includegraphics[width=#2\linewidth]{#1}
        \caption{#3} \label{#4}
    \end{figure}
}
\newcommand{\addimghere}[4]{ % Добавить рисунок непосредственно в это место
    \begin{figure}[H]
        \centering
        \includegraphics[width=#2\linewidth]{#1}
        \caption{#3} \label{#4}
    \end{figure}
}
\newcommand{\addtwoimghere}[5]{ % Вставка двух рисунков
    \begin{figure}[H]
        \centering
        \includegraphics[width=#2\linewidth]{#1}
        \hfill
        \includegraphics[width=#3\linewidth]{#2}
        \caption{#4} \label{#5}
    \end{figure}
}

% Заголовки секций в оглавлении в верхнем регистре
\usepackage{textcase}
\makeatletter
\let\oldcontentsline\contentsline
\def\contentsline#1#2{
    \expandafter\ifx\csname l@#1\endcsname\l@section
        \expandafter\@firstoftwo
    \else
        \expandafter\@secondoftwo
    \fi
    {\oldcontentsline{#1}{\MakeTextUppercase{#2}}}
    {\oldcontentsline{#1}{#2}}
}
\makeatother

% Оформление заголовков
\usepackage[compact,explicit]{titlesec}
\titleformat{\section}{}{}{12.5mm}{\centering{\thesection\quad\MakeTextUppercase{#1}}\vspace{1.5em}}
\titleformat{\subsection}[block]{\vspace{1em}}{}{12.5mm}{\thesubsection\quad#1\vspace{1em}}
\titleformat{\subsubsection}[block]{\vspace{1em}\normalsize}{}{12.5mm}{\thesubsubsection\quad#1\vspace{1em}}
\titleformat{\paragraph}[block]{\normalsize}{}{12.5mm}{#1}%\MakeTextUppercase{#1}}

% Секции без номеров (введение, заключение...), вместо section*{}
\newcommand{\anonsection}[1]{
    \phantomsection % Корректный переход по ссылкам в содержании
    \paragraph{\centerline{{#1}}\vspace{1.5em}}
    \addcontentsline{toc}{section}{\uppercase{#1}}
}

% Подсекции без номеров
\newcommand{\anonsubsection}[1]{
    \phantomsection
    \paragraph{\quad{#1}}
    \addcontentsline{toc}{subsection}{{#1}}
}

% Секции для приложений
\newcommand{\appsection}[1]{
    \phantomsection
    \paragraph{\centerline{{#1}}}
    \addcontentsline{toc}{section}{\uppercase{#1}}
}

% Библиография: отступы и межстрочный интервал
\makeatletter
\renewenvironment{thebibliography}[1]
    {\section*{\refname}
        \list{\@biblabel{\@arabic\c@enumiv}}
           {\settowidth\labelwidth{\@biblabel{#1}}
            \leftmargin\labelsep
            \itemindent 16.7mm
            \@openbib@code
            \usecounter{enumiv}
            \let\p@enumiv\@empty
            \renewcommand\theenumiv{\@arabic\c@enumiv}
        }
        \setlength{\itemsep}{0pt}
    }
\makeatother

\setcounter{page}{2} % Начало нумерации страниц
 % Подключаем преамбулу
\usepackage{tabularx}
\usepackage{makecell}


%%% Начало документа
\begin{document}

	% ТИТУЛЬНЫЙ ЛИСТ
	\begin{titlepage}
    \begin{center}
        \linespread{1.4}
    
        \normalsize{Министерство науки и высшего образования\\ Российской Федерации}\\
        \vspace{0.25cm}
        \normalsize{Федеральное государственное бюджетное образовательное\\ учреждение высшего образования}\\
        \vspace{0.25cm}
        \normalsize\textbf{«МОСКОВСКИЙ АВИАЦИОННЫЙ ИНСТИТУТ»}\\ {(НАЦИОНАЛЬНЫЙ ИССЛЕДОВАТЕЛЬСКИЙ УНИВЕРСИТЕТ)}\\
        \noindent\rule{\textwidth}{0.4pt} \\ \vspace{0.25cm}
        \normalsize
        {Институт № \InstitutNumber \ «\InstitutName»\\ Кафедра \KafedraNumber \ «\KafedraName»}\\
        \vfill
        
        {\RabotaType}\\
        {\RabotaOChem}\\
        \hfill\break	
        {\RabotaThemeName}\\
        
        \vfill	
        
        \begin{tabularx}{0.95\textwidth}{ l X l l }
            
            Выполнил & \makecell[c]{\underline{\hspace{3cm}}} & \makecell[l]{студент \StudentKurs \ курса \\ группы \StudentGroup} & \makecell[r]{\StudentFioLastname \ \StudentFioFirstname \\ \StudentFioSurname} \\
            
            \multicolumn{4}{ c }{ } \\
            
            Проверил & \makecell[c]{\underline{\hspace{3cm}}} & \makecell[l]{\PrepodCaptionFirst \\ \PrepodCaptionSecond} & \makecell[r]{\PrepodFioLastname \ \PrepodFioFirstname \\ \PrepodFioSurname} \\
            
        \end{tabularx}
    
    \end{center}
    
    \hfill \break
    \begin{center} Москва \the\year{} \end{center}
    % \begin{center} Москва 2022 \end{center}
    \thispagestyle{empty} % выключаем отображение номера для этой страницы
\end{titlepage}

\ifx \EmptyPageAfterTitle \undefined
\else
    \newpage
    \thispagestyle{empty}
    \mbox{}
    \newpage
\fi

%\newpagel


	% СОДЕРЖАНИЕ 
	\tableofcontents 
	\clearpage
 
        % ОСНОВНАЯ ЧАСТЬ 
	\section{Постановка задачи}

Дана расчетная схема гидромагистрали, представленная на рисунке \ref{fig:1-problem-scheme}. 
	\section{Решение}

\subsection{Решение стационарной задачи}

Для решения динамической задачи необходимо получить некоторые параметры потока на стационарном режиме. Для этого нужно решить стационарную задачу. Запишем её в виде системы уравнений
\begin{equation}
    \begin{cases}
        \dot{m}_{1_0} + \dot{m}_{2_0} - \dot{m}_{3_0} = 0, \\
        p_{4_0}-p_{\text{г}_0} = 0, \\
        p_{1_0} - p_{4_0} - \xi_1\dot{m}_{1_0}^2 = 0, \\
        p_2 - p_{4_0} - \xi_2\dot{m}_{2_0}^2 = 0, \\
        p_{4_0} - p_3 - \xi_3\dot{m}_{3_0}^2 = 0. 
    \end{cases}
\end{equation}

Решив систему уравнений, получим
\begin{equation}
    \begin{cases}
        \dot{m}_{1_0} = 10.515, \\
        \dot{m}_{2_0} = 3.249, \\
        \dot{m}_{3_0} = 13.764, \\
        p_{4_0} = 289442.719, \\
        p_{\ig_0} = 289442.719.
    \end{cases}
\end{equation}

Посчитаем массу газа по уравнению состояния. Запишем уравнение состояния газа
\begin{equation}
    \rho_{\ig} = \frac{p_{\ig_0}}{R_{\ig} \cdot T_{\ig}} = 0.476,
\end{equation}
где $R_{\ig} = \frac{R}{M_{\ig}} = \frac{8.314}{0.004026} = 2077.149$, а масса газа тогда будет равна
\begin{equation}
    m_{\ig} = \rho_{\ig} \cdot V_{\ig_0} = 0.001 \dkg.
\end{equation}

Решение стационарной задачи было произведено численным методом. Программный код расчета, написанный на языке программирования python, представлен в приложении.


\subsection{Решение динамической задачи с демпфером}
\subsubsection{Определение математической модели}

Запишем математическую модель для динамической системы с демпфером в виде следующей системы дифференциальных уравнений

\begin{equation}
    \begin{cases}
        \frac{d}{dt} p_4 = \frac{\left(\dot{m}_{1}+\dot{m}_{2}-\dot{m}_{3}-\dot{m}_{\id}\right)}{\frac{{V}_{\text{ж}_0} + {V}_{\id}}{{c}^{2}}}, \\
        
        \frac{d}{dt}V_{\id} = \frac{\dot{m}_{\id}}{\rho_{\text{ж}}}, \\
        
        \frac{d}{dt}m_{\id} = \frac{\left[p_{4}-\frac{m_{\ig}}{V_{\ig_0}-V_{\id}} \cdot R \cdot \left[T_{\ig_0} \cdot \left(\frac{p_{4}}{p_{4_0}}\right)^{\frac{k-1}{k}}\right]-\xi_{\id} \cdot \dot{m}_{\id} \cdot \left| \dot{m}_{\id}\right|\right]}{j_{\id}}, \\
        
        \frac{{d}}{{dt}}m_1 = \frac{\left({p}_1-{p}_4-{\xi}_1\cdot\dot{m}_1\cdot \left|\dot{m}_1 \right|\right)}{{j}_1}, \\
        
        {\frac{d}{dt}}m_{2} = {\frac{\left(p_{2}-p_{4}-\xi_{2}\cdot \dot{m}_{2}\cdot \left|\dot{m}_2 \right|\right)}{j_{2}}}, \\
        
        \frac{d}{dt}m_{3} = \frac{\left({p}_{4}-{p}_{3}-{\xi}_{3}\cdot \dot{m}_{3}\cdot \left|\dot{m}_3 \right|\right)}{{j}_{3}}.
    \end{cases}
    \label{eq:sys}
\end{equation}

Зададим закон импульсного возмущения для давления во второй магистрали $p_2$ такого вида

\begin{equation}
    p_2(t)=\left[
        \begin{matrix}
            A\sin\left(z(t)-\frac{\pi}{2}\right)+A+x_0 \quad &\text{если}\quad &0\leq t \leq T, \\
            x_0 \quad &\text{если}\quad &t\ >\ T,
        \end{matrix}
    \right.
    \label{eq:p2t}
\end{equation}
где
$z(t) = 2 \cdot t \cdot \frac{\pi}{T}$, $A = 0.1$ МПа - амплитуда возмущения, $T = 0.1$ секунды - период колебаний, $t = 0..1$ с шагом $h = 10^{-6}$ секунд - временная область, в которой исследуется система.

Подставляя (\ref{eq:p2t}) в систему дифференциальных уравнений (\ref{eq:sys}), решаем ее численным методом во временной области $t$. По получившимся результатам вычислений строим отображающие переходный процесс графики изменения величин $p_4$, $V_{\text{д}}$, $\dot{m}_1$, $\dot{m}_2$, $\dot{m}_3$ и $\dot{m}_{\text{д}}$ от времени $t$. Графики представлены ниже, а программный код расчета, написанный на языке программирования python, представлен в приложении.

\addimghere{inc/images/sys/p4}{0.88}{ График зависимости $p_4(t)$ }{graf:sys-p4}
\addimghere{inc/images/sys/pg}{0.88}{ График зависимости $p_{\ig}(t)$ }{graf:sys-pg}
\addimghere{inc/images/sys/Vd}{0.88}{ График зависимости $V_{\id}(t)$ }{graf:sys-Vd}
\addimghere{inc/images/sys/md}{0.88}{ График зависимости $\dot{m}_{\id}(t)$ }{graf:sys-md}
\addimghere{inc/images/sys/m1}{0.88}{ График зависимости $\dot{m}_1(t)$ }{graf:sys-m1}
\addimghere{inc/images/sys/m2}{0.88}{ График зависимости $\dot{m}_2(t)$ }{graf:sys-m2}
\addimghere{inc/images/sys/m3}{0.88}{ График зависимости $\dot{m}_3(t)$ }{graf:sys-m3}
\addimghere{inc/images/sys/p2}{0.88}{ График зависимости $p_2(t)$ }{graf:sys-p2}


\subsubsection{Определение АЧХ системы}

Для построения амплитудно-частотной характеристики необходимо знать амплитуду колебаний системы при различных частотах возмущения. Зададим синусоидальное возмущение $p_2(t, f)$ таким образом
\begin{equation}
    p_2(t, f) = f \cdot 2\pi \cdot t,
    \label{eq:p2tf}
\end{equation}
где $f$ - частота колебаний $p_2$. 

Тогда АЧХ будем искать в виде
\begin{equation}
    \textit{АЧХ} = \frac{\frac{max(p_4)-min(p_4)}{2}}{A},
    \label{eq:afc}
\end{equation}
где максимальные и минимальные значения будем искать на утсановившемся режиме в диапазоне времени от 0.7 до 0.8 секунд. 

Подставляя (\ref{eq:p2tf}) в систему уравнений (\ref{eq:sys}), находим решения для каждого значения частоты $f = 0..1000$ Гц с увеличивающимся шагом.

График реакции $p_4$ на синусоидальное возмущение $p_2$, заданное в (\ref{eq:p2tf}), при частоте $f = 15$ и возмущения $p_2$ показаны на рисунке (\ref{graf:sys-afc-p4}) и (\ref{graf:sys-afc-p2}) соответственно. Амплитудно-частотная характеристика системы показана на рисунке (\ref{graf:sys-afc-afc}). Программный код расчета, написанный на языке программирования python, представлен в приложении.

\addimghere{inc/images/sys-afc/p4}{0.88}{ График зависимости $p_4(t)$ при $f = 15$ Гц }{graf:sys-afc-p4}
\addimghere{inc/images/sys-afc/p2}{0.88}{ График зависимости $p_2(t)$ при $f = 15$ Гц }{graf:sys-afc-p2}
\addimghere{inc/images/sys-afc/afc}{0.88}{ Амплитудно-частотная характеристика }{graf:sys-afc-afc}


\subsection{Решение динамической задачи без демпфера}
\subsubsection{Определение математической модели}

Запишем математическую модель для динамической системы без демпфера в виде следующей системы дифференциальных уравнений

\begin{equation}
    \begin{cases}
        \frac{d}{dt} p_4 = \frac{\left(\dot{m}_{1}+\dot{m}_{2}-\dot{m}_{3}\right)}{\frac{{V}_{\text{ж}_0}}{{c}^{2}}}, \\
        
        \frac{{d}}{{dt}}m_1 = \frac{\left({p}_1-{p}_4-{\xi}_1\cdot\dot{m}_1\cdot \left|\dot{m}_1 \right|\right)}{{j}_1}, \\
        
        {\frac{d}{dt}}m_{2} = {\frac{\left(p_{2}-p_{4}-\xi_{2}\cdot \dot{m}_{2}\cdot \left|\dot{m}_2 \right|\right)}{j_{2}}}, \\
        
        \frac{d}{dt}m_{3} = \frac{\left({p}_{4}-{p}_{3}-{\xi}_{3}\cdot \dot{m}_{3}\cdot \left|\dot{m}_3 \right|\right)}{{j}_{3}}.
    \end{cases}
    \label{eq:sys-nd}
\end{equation}

Зададим закон импульсного возмущения для давления во второй магистрали $p_2$ такого же вида, как (\ref{eq:p2t}).

Подставляя (\ref{eq:p2t}) в систему дифференциальных уравнений (\ref{eq:sys-nd}), решаем ее численным методом во временной области $t$. По получившимся результатам вычислений строим отображающие переходный процесс графики изменения величин $p_4$, $V_{\id}$, $\dot{m}_1$, $\dot{m}_2$, $\dot{m}_3$ и $\dot{m}_{\id}$ от времени $t$. Графики представлены ниже, а программный код расчета, написанный на языке программирования python, представлен в приложении.

\addimghere{inc/images/sys-nd/p4}{0.88}{ График зависимости $p_4(t)$ }{graf:sys-nd-p4}
\addimghere{inc/images/sys-nd/m1}{0.88}{ График зависимости $\dot{m}_1(t)$ }{graf:sys-nd-m1}
\addimghere{inc/images/sys-nd/m2}{0.84}{ График зависимости $\dot{m}_2(t)$ }{graf:sys-nd-m2}
\addimghere{inc/images/sys-nd/m3}{0.88}{ График зависимости $\dot{m}_3(t)$ }{graf:sys-nd-m3}


\subsubsection{Определение АЧХ системы}

Для построения амплитудно-частотной характеристики необходимо знать амплитуду колебаний системы при различных частотах возмущения. Зададим синусоидальное возмущение $p_2(t, f)$ таким же образом, как и в (\ref{eq:p2tf}). АЧХ будем искать в виде (\ref{eq:afc}), где максимальные и минимальные значения будем искать на утсановившемся режиме в диапазоне времени от 0.7 до 0.8 секунд. 

Подставляя (\ref{eq:p2tf}) в систему уравнений (\ref{eq:sys-nd}), находим решения для каждого значения частоты $f = 0..1000$ Гц с увеличивающимся шагом.

График реакции $p_4$ на синусоидальное возмущение $p_2$, заданное в (\ref{eq:p2tf}), при частоте $f = 15$ и возмущения $p_2$ показаны на рисунке (\ref{graf:sys-afc-nd-p4}) и (\ref{graf:sys-afc-nd-p2}) соответственно. Амплитудно-частотная характеристика системы показана на рисунке (\ref{graf:sys-afc-nd-afc}). Программный код расчета, написанный на языке программирования python, представлен в приложении.

\addimghere{inc/images/sys-nd-afc/p4}{0.88}{ График зависимости $p_4(t)$ при $f = 15$ Гц }{graf:sys-afc-nd-p4}
\addimghere{inc/images/sys-nd-afc/p2}{0.88}{ График зависимости $p_2(t)$ при $f = 15$ Гц }{graf:sys-afc-nd-p2}
\addimghere{inc/images/sys-nd-afc/afc}{0.88}{ Амплитудно-частотная характеристика }{graf:sys-afc-nd-afc}
	\section{Заключение}

В данной работе была составлена математическая модель гидромагистрали с демпфером и без демпфера, произведен расчет системы дифференциальных уравнений, описывающих процессы в колебательной системе мат. модели, с помощью численных методов и получены решения для искомых по заданию величин  $p_4$, $V_{\text{д}}$, $\dot{m}_1$, $\dot{m}_2$, $\dot{m}_3$ и $\dot{m}_{\text{д}}$ и др. Так же были построены амплитудно-частотные характеристики для систем.

Взглянув на полученные графики, отображающие переходные процессы в системе с демпфером и без него (рисунки (\ref{graf:sys-p4}), (\ref{graf:sys-nd-p4}) соответственно), можно заметить, что демпфер сработал как и ожидалось, то есть уменьшил амплитуду вынужденных колебаний системы.

Все вычисления производились на ЭВМ, программный код на языке программирования python прилагается. Для расчета систем дифференциальных уравнений использовался модуль scipy, метод решения - LSODA \cite{metod}.

	% БИБЛИОГРАФИЧЕСКИЙ СПИСОК
	\begingroup 
\renewcommand{\section}[2]{\anonsection{\uppercase{Библиографический список}}}
\begin{thebibliography}{00}

\bibitem{}
    Попов Е.П.
    Теория линейных систем автоматического регулирования и управления [Текст] : учеб. пособие для втузов / Е.П. Попов. - 2-е изд., перераб. и доп. - М. : Наука, 1989. - 301 с. 

\bibitem{metod}
    SciPy v1.11.4 Manual
    [Электронный ресурс] // 
    URL: https://docs.scipy.org/doc/scipy/reference/generated/scipy.integrate.odeint.html
    (дата обращения: 10.декабря.2023)

\end{thebibliography}
\endgroup

\clearpage


	% ПРИЛОЖЕНИЯ
	\vspace*{\fill}
\centering{\uppercase{Приложение}}
\vspace*{\fill}

\clearpage

% -- А --------------------------------------------------------------------------------
\appsection{Приложение А}
\centering{\uppercase{Рабочий чертёж детали}}
\vspace{\baselineskip}
% \clearpage

% \eject \pdfpagewidth=420mm \pdfpageheight=297mm
% \newgeometry{
%     layoutwidth=420mm,
%     layoutheight=297mm,
%     top=-0.1mm,
%     left=-1mm,
%     right=-1mm,
%     bottom=-1mm
%     }
% \includegraphics[scale=0.6]{inc/images/drawing1}
% \includegraphics[width=0.997\textwidth]{inc/images/drawing1-1}
% \thispagestyle{empty}

\eject \pdfpagewidth=297mm \pdfpageheight=210mm
\newgeometry{
    layoutwidth=297mm,
    layoutheight=210mm,
    top=-0.1mm,
    left=-1mm,
    right=-1mm,
    bottom=-1mm
}
\includegraphics[width=0.9957\textwidth]{inc/images/drawing1-1}
\thispagestyle{empty}


% -- Б --------------------------------------------------------------------------------
\eject \pdfpagewidth=210mm \pdfpageheight=297mm
\restoregeometry
\appsection{Приложение Б}
\centering{\uppercase{Чертёж заготовки}}
\vspace{\baselineskip}

\newgeometry{
    layoutwidth=210mm,
    layoutheight=297mm,
    top=-0.1mm,
    left=-1mm,
    right=-1mm,
    bottom=-1mm
}
\includegraphics[width=0.993457\textwidth]{inc/images/drawing2}

\restoregeometry

\clearpage
	% \includepdf{images/drawing1.pdf}


	% \hfill \break
	% \begin{center}
	% 	\LaTeX
	% \end{center}

\end{document}
%%% Конец документа