\section{Решение}

\subsection{Решение стационарной задачи}

Для решения динамической задачи необходимо получить некоторые параметры потока на стационарном режиме. Для этого нужно решить стационарную задачу. Запишем её в виде системы уравнений
\begin{equation}
    \begin{cases}
        \dot{m}_{1_0} + \dot{m}_{2_0} - \dot{m}_{3_0} = 0, \\
        p_{4_0}-p_{\text{г}_0} = 0, \\
        p_{1_0} - p_{4_0} - \xi_1\dot{m}_{1_0}^2 = 0, \\
        p_2 - p_{4_0} - \xi_2\dot{m}_{2_0}^2 = 0, \\
        p_{4_0} - p_3 - \xi_3\dot{m}_{3_0}^2 = 0. 
    \end{cases}
\end{equation}

Решив систему уравнений, получим
\begin{equation}
    \begin{cases}
        \dot{m}_{1_0} = 10.515, \\
        \dot{m}_{2_0} = 3.249, \\
        \dot{m}_{3_0} = 13.764, \\
        p_{4_0} = 289442.719, \\
        p_{\ig_0} = 289442.719.
    \end{cases}
\end{equation}

Посчитаем массу газа по уравнению состояния. Запишем уравнение состояния газа
\begin{equation}
    \rho_{\ig} = \frac{p_{\ig_0}}{R_{\ig} \cdot T_{\ig}} = 0.476,
\end{equation}
где $R_{\ig} = \frac{R}{M_{\ig}} = \frac{8.314}{0.004026} = 2077.149$, а масса газа тогда будет равна
\begin{equation}
    m_{\ig} = \rho_{\ig} \cdot V_{\ig_0} = 0.001 \dkg.
\end{equation}

Решение стационарной задачи было произведено численным методом. Программный код расчета, написанный на языке программирования python, представлен в приложении.


\subsection{Решение динамической задачи с демпфером}
\subsubsection{Определение математической модели}

Запишем математическую модель для динамической системы с демпфером в виде следующей системы дифференциальных уравнений

\begin{equation}
    \begin{cases}
        \frac{d}{dt} p_4 = \frac{\left(\dot{m}_{1}+\dot{m}_{2}-\dot{m}_{3}-\dot{m}_{\id}\right)}{\frac{{V}_{\text{ж}_0} + {V}_{\id}}{{c}^{2}}}, \\
        
        \frac{d}{dt}V_{\id} = \frac{\dot{m}_{\id}}{\rho_{\text{ж}}}, \\
        
        \frac{d}{dt}m_{\id} = \frac{\left[p_{4}-\frac{m_{\ig}}{V_{\ig_0}-V_{\id}} \cdot R \cdot \left[T_{\ig_0} \cdot \left(\frac{p_{4}}{p_{4_0}}\right)^{\frac{k-1}{k}}\right]-\xi_{\id} \cdot \dot{m}_{\id} \cdot \left| \dot{m}_{\id}\right|\right]}{j_{\id}}, \\
        
        \frac{{d}}{{dt}}m_1 = \frac{\left({p}_1-{p}_4-{\xi}_1\cdot\dot{m}_1\cdot \left|\dot{m}_1 \right|\right)}{{j}_1}, \\
        
        {\frac{d}{dt}}m_{2} = {\frac{\left(p_{2}-p_{4}-\xi_{2}\cdot \dot{m}_{2}\cdot \left|\dot{m}_2 \right|\right)}{j_{2}}}, \\
        
        \frac{d}{dt}m_{3} = \frac{\left({p}_{4}-{p}_{3}-{\xi}_{3}\cdot \dot{m}_{3}\cdot \left|\dot{m}_3 \right|\right)}{{j}_{3}}.
    \end{cases}
    \label{eq:sys}
\end{equation}

Зададим закон импульсного возмущения для давления во второй магистрали $p_2$ такого вида

\begin{equation}
    p_2(t)=\left[
        \begin{matrix}
            A\sin\left(z(t)-\frac{\pi}{2}\right)+A+x_0 \quad &\text{если}\quad &0\leq t \leq T, \\
            x_0 \quad &\text{если}\quad &t\ >\ T,
        \end{matrix}
    \right.
    \label{eq:p2t}
\end{equation}
где
$z(t) = 2 \cdot t \cdot \frac{\pi}{T}$, $A = 0.1$ МПа - амплитуда возмущения, $T = 0.1$ секунды - период колебаний, $t = 0..1$ с шагом $h = 10^{-6}$ секунд - временная область, в которой исследуется система.

Подставляя (\ref{eq:p2t}) в систему дифференциальных уравнений (\ref{eq:sys}), решаем ее численным методом во временной области $t$. По получившимся результатам вычислений строим отображающие переходный процесс графики изменения величин $p_4$, $V_{\text{д}}$, $\dot{m}_1$, $\dot{m}_2$, $\dot{m}_3$ и $\dot{m}_{\text{д}}$ от времени $t$. Графики представлены ниже, а программный код расчета, написанный на языке программирования python, представлен в приложении.

\addimghere{inc/images/sys/p4}{0.88}{ График зависимости $p_4(t)$ }{graf:sys-p4}
\addimghere{inc/images/sys/pg}{0.88}{ График зависимости $p_{\ig}(t)$ }{graf:sys-pg}
\addimghere{inc/images/sys/Vd}{0.88}{ График зависимости $V_{\id}(t)$ }{graf:sys-Vd}
\addimghere{inc/images/sys/md}{0.88}{ График зависимости $\dot{m}_{\id}(t)$ }{graf:sys-md}
\addimghere{inc/images/sys/m1}{0.88}{ График зависимости $\dot{m}_1(t)$ }{graf:sys-m1}
\addimghere{inc/images/sys/m2}{0.88}{ График зависимости $\dot{m}_2(t)$ }{graf:sys-m2}
\addimghere{inc/images/sys/m3}{0.88}{ График зависимости $\dot{m}_3(t)$ }{graf:sys-m3}
\addimghere{inc/images/sys/p2}{0.88}{ График зависимости $p_2(t)$ }{graf:sys-p2}


\subsubsection{Определение АЧХ системы}

Для построения амплитудно-частотной характеристики необходимо знать амплитуду колебаний системы при различных частотах возмущения. Зададим синусоидальное возмущение $p_2(t, f)$ таким образом
\begin{equation}
    p_2(t, f) = f \cdot 2\pi \cdot t,
    \label{eq:p2tf}
\end{equation}
где $f$ - частота колебаний $p_2$. 

Тогда АЧХ будем искать в виде
\begin{equation}
    \textit{АЧХ} = \frac{\frac{max(p_4)-min(p_4)}{2}}{A},
    \label{eq:afc}
\end{equation}
где максимальные и минимальные значения будем искать на утсановившемся режиме в диапазоне времени от 0.7 до 0.8 секунд. 

Подставляя (\ref{eq:p2tf}) в систему уравнений (\ref{eq:sys}), находим решения для каждого значения частоты $f = 0..1000$ Гц с увеличивающимся шагом.

График реакции $p_4$ на синусоидальное возмущение $p_2$, заданное в (\ref{eq:p2tf}), при частоте $f = 15$ и возмущения $p_2$ показаны на рисунке (\ref{graf:sys-afc-p4}) и (\ref{graf:sys-afc-p2}) соответственно. Амплитудно-частотная характеристика системы показана на рисунке (\ref{graf:sys-afc-afc}). Программный код расчета, написанный на языке программирования python, представлен в приложении.

\addimghere{inc/images/sys-afc/p4}{0.88}{ График зависимости $p_4(t)$ при $f = 15$ Гц }{graf:sys-afc-p4}
\addimghere{inc/images/sys-afc/p2}{0.88}{ График зависимости $p_2(t)$ при $f = 15$ Гц }{graf:sys-afc-p2}
\addimghere{inc/images/sys-afc/afc}{0.88}{ Амплитудно-частотная характеристика }{graf:sys-afc-afc}


\subsection{Решение динамической задачи без демпфера}
\subsubsection{Определение математической модели}

Запишем математическую модель для динамической системы без демпфера в виде следующей системы дифференциальных уравнений

\begin{equation}
    \begin{cases}
        \frac{d}{dt} p_4 = \frac{\left(\dot{m}_{1}+\dot{m}_{2}-\dot{m}_{3}\right)}{\frac{{V}_{\text{ж}_0}}{{c}^{2}}}, \\
        
        \frac{{d}}{{dt}}m_1 = \frac{\left({p}_1-{p}_4-{\xi}_1\cdot\dot{m}_1\cdot \left|\dot{m}_1 \right|\right)}{{j}_1}, \\
        
        {\frac{d}{dt}}m_{2} = {\frac{\left(p_{2}-p_{4}-\xi_{2}\cdot \dot{m}_{2}\cdot \left|\dot{m}_2 \right|\right)}{j_{2}}}, \\
        
        \frac{d}{dt}m_{3} = \frac{\left({p}_{4}-{p}_{3}-{\xi}_{3}\cdot \dot{m}_{3}\cdot \left|\dot{m}_3 \right|\right)}{{j}_{3}}.
    \end{cases}
    \label{eq:sys-nd}
\end{equation}

Зададим закон импульсного возмущения для давления во второй магистрали $p_2$ такого же вида, как (\ref{eq:p2t}).

Подставляя (\ref{eq:p2t}) в систему дифференциальных уравнений (\ref{eq:sys-nd}), решаем ее численным методом во временной области $t$. По получившимся результатам вычислений строим отображающие переходный процесс графики изменения величин $p_4$, $V_{\id}$, $\dot{m}_1$, $\dot{m}_2$, $\dot{m}_3$ и $\dot{m}_{\id}$ от времени $t$. Графики представлены ниже, а программный код расчета, написанный на языке программирования python, представлен в приложении.

\addimghere{inc/images/sys-nd/p4}{0.88}{ График зависимости $p_4(t)$ }{graf:sys-nd-p4}
\addimghere{inc/images/sys-nd/m1}{0.88}{ График зависимости $\dot{m}_1(t)$ }{graf:sys-nd-m1}
\addimghere{inc/images/sys-nd/m2}{0.84}{ График зависимости $\dot{m}_2(t)$ }{graf:sys-nd-m2}
\addimghere{inc/images/sys-nd/m3}{0.88}{ График зависимости $\dot{m}_3(t)$ }{graf:sys-nd-m3}


\subsubsection{Определение АЧХ системы}

Для построения амплитудно-частотной характеристики необходимо знать амплитуду колебаний системы при различных частотах возмущения. Зададим синусоидальное возмущение $p_2(t, f)$ таким же образом, как и в (\ref{eq:p2tf}). АЧХ будем искать в виде (\ref{eq:afc}), где максимальные и минимальные значения будем искать на утсановившемся режиме в диапазоне времени от 0.7 до 0.8 секунд. 

Подставляя (\ref{eq:p2tf}) в систему уравнений (\ref{eq:sys-nd}), находим решения для каждого значения частоты $f = 0..1000$ Гц с увеличивающимся шагом.

График реакции $p_4$ на синусоидальное возмущение $p_2$, заданное в (\ref{eq:p2tf}), при частоте $f = 15$ и возмущения $p_2$ показаны на рисунке (\ref{graf:sys-afc-nd-p4}) и (\ref{graf:sys-afc-nd-p2}) соответственно. Амплитудно-частотная характеристика системы показана на рисунке (\ref{graf:sys-afc-nd-afc}). Программный код расчета, написанный на языке программирования python, представлен в приложении.

\addimghere{inc/images/sys-nd-afc/p4}{0.88}{ График зависимости $p_4(t)$ при $f = 15$ Гц }{graf:sys-afc-nd-p4}
\addimghere{inc/images/sys-nd-afc/p2}{0.88}{ График зависимости $p_2(t)$ при $f = 15$ Гц }{graf:sys-afc-nd-p2}
\addimghere{inc/images/sys-nd-afc/afc}{0.88}{ Амплитудно-частотная характеристика }{graf:sys-afc-nd-afc}