\section{Метод изготовления заготовки}

Проектировка заготовки из материала Сталь-45 для изготовления детали осуществляется по следующему плану. 

Деталь получается путем токарной обработки по всему контуру, а затем – фрезерной обработки. Необрабатываемых поверхностей в детали нет. По конфигурации деталь имеет цилиндрическую форму. В этой связи наиболее удобный метод получения заготовки из круглого горячекатаного прутка (ГОСТ 2590-2006). Выбираем прокат обычной точности.

Диаметр описанной окружности детали $d = 150$ мм, тогда предельные отклонения для детали будут:

\begin{equation}
    es_\text{д} = 0    \text{ мм}
\end{equation}
\begin{equation}
    ei_\text{д} = -1    \text{ мм}
\end{equation}

Наименьший операционный припуск $Z_{min}$ складывается из отдельных элементов:
\begin{equation}
    Z_\text{min} = Z_1 + Z_2 + Z_3 = 1.35   \text{,}
\end{equation}
где $Z_1$ - слой материала, который необходимо удалить с заготовки для устранения неровностей $R_z$ и дефектного слоя $h$:
\begin{equation}
    Z_1 = R_z + h = 150 + 500 = 650 \text{ мкм,}
\end{equation}
$Z_2 = 0.6 \text{ мм}$ - слой, удаляемый для компенсации погрешностей формы
Номинальный размер заготовки, $Z_3 = 0.1 \text{ мм}$ - слой, удаляемый для компенсации погрешностей установки.\cite{metodika}

Диаметр заготовки \cite{metodika}:
\begin{equation}
    d_\text{з} = d_\text{д} + 2 \cdot Z_{min} + 2 \cdot ei_\text{д} = 150 + 2 \cdot 1.35 + 2 \cdot (-1) = 154.7    \text{ мм}
\end{equation}

Ближайшее значение диаметра прутка по ГОСТ 2590-2006:
\begin{equation}
    d_\text{з} =  155^{+0.8}_{-2.0}   \text{ мм}
\end{equation}

Номинальный припуск на обработку \cite{metodika}:
\begin{equation}
    Z_\text{ном} = \dfrac{d_\text{з} - d_\text{д}}{2} = \dfrac{155 - 150}{2} = 2.5   \text{ мм}
\end{equation}

Максимальный припуск на обработку \cite{metodika}:
\begin{equation}
    Z_\text{max} = Z_\text{min} + ei_\text{д} + es_\text{з} = 2.5 + 1 + 0.8 = 4.3   \text{ мм}
\end{equation}

Припуск по торцам равен номинальному припуску на диаметр, тогда длина заготовки \cite{metodika}:
\begin{equation}
    L = L_\text{д} + 2 \cdot Z_\text{ном} = 60 + 2 \cdot 2.5 = 65   \text{ мм}
\end{equation}

% \clearpage