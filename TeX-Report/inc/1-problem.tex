\section{Постановка задачи}

Дана расчетная схема гидромагистрали, представленная на рисунке \ref{fig:1-problem-scheme}. Гидромагистраль представляет собой три трубопровода, по которым течет жидкость, и соединяющую их полость с установленным в ней демпфером. За демпфер принимается полость с газом, отделенным от жидкости идеальной мембраной.

\addimghere{inc/images/1-problem-scheme}{0.8}{ Расчетная схема гидромагистрали }{fig:1-problem-scheme}


\subsection{Задание} 

Необходимо составить математическую модель работы гидромагистрали, и на ее основе решить следующие задачи:
\begin{enumerate}
    \item задавшись возмущением $p_2$, изучить его влияние в гидромагистрали с демпфером на величины $p_4$, $V_{\text{д}}$, $\dot{m}_1$, $\dot{m}_2$, $\dot{m}_3$ и $\dot{m}_{\text{д}}$;
    \item задавшись возмущением $p_2$, изучить его влияние в гидромагистрали без демпфера на величины $p_4$, $V_{\text{д}}$, $\dot{m}_1$, $\dot{m}_2$, $\dot{m}_3$ и $\dot{m}_{\text{д}}$;
    \item построить амплитудно-частотную характеристику $W = \frac{p_4}{p_2}$.
\end{enumerate}


\subsection{Исходные данные}

Исходные данные представлены в таблице \ref{tab:1-problem}, а в таблице \ref{tab:1-problem-2} представлены значения параметров системы.

\begin{table}[H]
    \caption{Исходные установки}
    \label{tab:1-problem}
    \centering
    \begin{tabular}{ |l|c| }
        \hline \textbf{Параметр} & \textbf{Значение} \\
        \hline Рабочая жидкость & Керосин \\
        \hline Демпфирующий газ & Гелий \\
        \hline Возмущение & Импульсное \\
        \hline
    \end{tabular}
\end{table}

\begin{table}[H]
    \caption{Параметры системы}
    \label{tab:1-problem-2}
    \centering
    \begin{tabular}{ |l|c|c|c| }
        \hline \textbf{Группа параметров} & \textbf{Параметр} & \textbf{Значение} & \textbf{Размерность} \\
        \hline \multirow{3}{*}{Давления в трубопроводах} & $p_{1}$ & 0.4 &  \\
        \cline{2-3} \multirow{3}{*}{} & $p_{2_0}$ & 0.3 & МПа \\
        \cline{2-3} \multirow{3}{*}{} & $p_{3}$ & 0.1 &  \\
        
        \hline \multirow{2}{*}{Коэффициенты инерционности} & $j_{1,2,3}$    & 400 & \multirow{2}{*}{ $\frac{1}{\text{м}}$ } \\
        \cline{2-3} \multirow{2}{*}{} & $j_{\text{д}}$ & 20  & \multirow{2}{*}{} \\

        \hline Коэффициенты гидросопротивления & $\xi_{1,2,3,\text{д}}$ & 1000 & $\frac{1}{\text{кг} \cdot \text{м}}$ \\

        \hline Объём проточной области & $V_{\text{ж}_0}$ & 0.004 & $\text{м}^3$ \\

        \hline Объём газа & $V_{\text{г}_0}$ & 0.003 & $\text{м}^3$ \\
        \hline Температура газа & $T_{\text{г}_0}$ & 293 & K \\
        \hline
    \end{tabular}
\end{table}