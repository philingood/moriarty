\section{Характеристика обрабатываемого материала}

\subsection{Химический состав}

Обрабатываемый материал – Сталь-45. Химический состав данного материала в процентном соотношении в соответствии с ГОСТ 1050-2013 приведен в таблице \ref{tab:nigger}.

\begin{table}[H]
    \centering
    \caption[]{Химический состав материала Сталь-45 в процентном соотношении (ГОСТ 1050-2013)}
    \label{tab:nigger}
    \begin{tabular}{|p{1.5cm}|p{1.5cm}|p{1.5cm}|p{1.5cm}|p{1.5cm}|p{1.5cm}|p{1.5cm}|p{1.5cm}|}
        \hline \textbf{$C$} & \textbf{$Mn$} & \textbf{$Si$} & \textbf{$Ni$} & \textbf{$Cr$} & \textbf{$Cu$} & \textbf{$S$} & \textbf{$P$} \\
        \hline 0,42—0,5\%   & 0,5—0,8\%     & 0,17—0,37\%   & до 0,3\%      & до 0,25\%     & до 0,3\%      & до 0,035\%    & до 0,03\% \\
        \hline
    \end{tabular}
\end{table}


\subsection{Физико-механические свойства}

Детали из стали марки 45 подвергаются нормализации при температуре 860-880° С или закалке в воде с температуры 840-860° С с последующим отпуском.\cite{site-stal-45}

Сталь горячекатаная согласно ГОСТ 1050-88 имеет предел прочности $\sigma_\textit{В}=600$ МПа с относительным удлинением $\delta=16\%$.


\subsection{Область применения}

Сталь-45 применяется для изготовления таких конструкционных элементов, как вал-шестерни, коленчатые и распределительные валы, шестерни, шпиндели, бандажи, цилиндры, кулачки и другие нормализованные, улучшаемые и подвергаемые поверхностной термообработке детали, от которых требуется повышенная прочность.\cite{site-stal-45}


\subsection{Технологические свойства}

Обрабатываемость резанием - $K_\textit{v тв.спл} = 1$ и $K_\textit{v б.ст} = 1$ в горячекатаном состоянии при $H\!B = 170 \div 179$ и $\sigma_\textit{в} = 640$ МПа.


% \clearpage