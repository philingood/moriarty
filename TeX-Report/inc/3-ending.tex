\section{Заключение}

В данной работе была составлена математическая модель гидромагистрали с демпфером и без демпфера, произведен расчет системы дифференциальных уравнений, описывающих процессы в колебательной системе мат. модели, с помощью численных методов и получены решения для искомых по заданию величин  $p_4$, $V_{\text{д}}$, $\dot{m}_1$, $\dot{m}_2$, $\dot{m}_3$ и $\dot{m}_{\text{д}}$ и др. Так же были построены амплитудно-частотные характеристики для систем.

Взглянув на полученные графики, отображающие переходные процессы в системе с демпфером и без него (рисунки (\ref{graf:sys-p4}), (\ref{graf:sys-nd-p4}) соответственно), можно заметить, что демпфер сработал как и ожидалось, то есть уменьшил амплитуду вынужденных колебаний системы.

Все вычисления производились на ЭВМ, программный код на языке программирования python прилагается. Для расчета систем дифференциальных уравнений использовался модуль scipy, метод решения - LSODA \cite{metod}.